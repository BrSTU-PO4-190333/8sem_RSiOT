\documentclass[12pt, a4paper, simple]{eskdtext}

\usepackage{config/main.env}
\usepackage{config/report.env}
\usepackage{styles/listing}
\usepackage{styles/lists}
\usepackage{styles/SectionMargins}
\usepackage{styles/table}
\usepackage{styles/TableOfContent}
\usepackage{styles/url}

\begin{document}
  \input{include/TitlePage.report.tex}

  % = = = = = = = = = = = = = = = =
  \ESKDstyle{empty}
  \begin{center}
    \textbf{Отчёт лабораторной работы №\envReportLabNumber}
  \end{center}

  \paragraph{} \textbf{Тема}: <<\envReportTitle>>

  \paragraph{} \textbf{Цель}:
  Знакомство с Kubernetes Engine, Container Registry.

  \paragraph{} \textbf{Что нужно сделать}:

  Полученные в предыдущей работе образы должны быть запущены в Google Kubernetes Engine.
  Для этого сохраните их в Google Container Registry.

  Для создания и управления кластером используйте kubectl.
  Настройте все взаимодействия между сервисами.

  После запуска убедитесь в BigQuery, Cloud Functions, что общая система продолжает функционировать,
  несмотря на переезд микросервисов из GCE и GAE в GKE.

  \paragraph{} \textbf{{Ход работы}}:

  Посмотрел видео на YouTube \cite{kubernetes_install} по установке VirtualBox, kubectl, minikube на Win10.

  \begin{lstlisting}[
    frame=single,
    rulecolor=\color{blue},
    name={Установка VirtualBox на Ubuntu 22.10},
  ]
sudo apt update
sudo apt install virtualbox
\end{lstlisting}

  Почитал официальную документацию \cite{kubernetes_install_kubectl} и установил kubectl.

\begin{lstlisting}[
  frame=single,
  rulecolor=\color{blue},
  name={Установка kubectl на Ubuntu 22.10},
]
curl -LO "https://dl.k8s.io/release/$(curl -L -s https://dl.k8s.io/release/stable.txt)/bin/linux/amd64/kubectl"
curl -LO "https://dl.k8s.io/$(curl -L -s https://dl.k8s.io/release/stable.txt)/bin/linux/amd64/kubectl.sha256"
echo "$(cat kubectl.sha256)  kubectl" | sha256sum --check
sudo install -o root -g root -m 0755 kubectl /usr/local/bin/kubectl
kubectl version --client
\end{lstlisting}

  Почитал официальную документацию \cite{kubernetes_install_minikube} и установил minikube.

\begin{lstlisting}[
  frame=single,
  rulecolor=\color{blue},
  name={Установка minikube на Ubuntu 22.10},
]
curl -LO https://storage.googleapis.com/minikube/releases/latest/minikube-linux-amd64
sudo install minikube-linux-amd64 /usr/local/bin/minikube
minikube version
\end{lstlisting}

  Посмотрел теорию на YouTube \cite{kubernetes_theory} по Kubernetes.

  Посмотрел практику по подам Kubernetes на YouTube \cite{kubernetes_create_pod} и создал конфиги pod-ms3-ver1.yaml, pod-ms2-ver1.yaml.

  \newpage
  \lstinputlisting[
    frame=single,
    rulecolor=\color{magenta},
  ]{../sources/K8S/pod-ms3-ver1.example.yaml}

  \begin{lstlisting}[
    frame=single,
    rulecolor=\color{blue},
    name={Запуск пода MS3},
  ]
cp pod-ms3-ver1.example.yaml pod-ms3-ver1.yaml

# меняем настройки env в файле pod-ms3-ver1.yaml

kubectl apply -f pod-ms3-ver1.yaml  # запускаем под

kubectl get pods                    # смотрим активные поды
kubectl describe pods ms3           # больше информации о MS3
kubectl logs ms3                    # смотрим логи MS3
kubectl port-forward ms3 3333:8080  # открываем порт на своей машине

kubectl get pods                    # смотрим активные поды
kubectl delete -f pod-ms3-ver1.yaml # удаляем под kubectl delete pods ms3
kubectl get pods                    # смотрим активные поды
\end{lstlisting}

  \newpage

  \lstinputlisting[
    frame=single,
    rulecolor=\color{magenta},
  ]{../sources/K8S/pod-ms2-ver1.example.yaml}

  \newpage

  \begin{lstlisting}[
    frame=single,
    rulecolor=\color{blue},
    name={Запуск пода MS2},
  ]
cp pod-ms2-ver1.example.yaml pod-ms2-ver1.yaml

# меняем настройки env в файле pod-ms2-ver1.yaml

kubectl apply -f pod-ms2-ver1.yaml  # запускаем под

kubectl get pods                    # смотрим активные поды
kubectl describe pods ms2           # больше информации о MS2
kubectl logs ms2                    # смотрим логи MS2
kubectl port-forward ms2 2222:44480 # открываем порт на своей машине

kubectl get pods                    # смотрим активные поды
kubectl delete -f pod-ms2-ver1.yaml # удаляем под kubectl delete pods ms2
kubectl get pods                    # смотрим активные поды
\end{lstlisting}

  \paragraph{} \textbf{Вывод}:
  Установили VirtualBox, kubectl, minikube.
  Создали кластер в VirtualBox используя minikube.
  Создали под для MS3 с помощью kubectl и конфига pod-ms3-ver1.yaml.
  Создали под для MS2 с помощью kubectl и конфига pod-ms2-ver1.yaml.

  % = = = = = = = = = = = = = = = =
  % \newpage
  \begingroup
    \phantomsection
    \addcontentsline{toc}{section}{СПИСОК ИСПОЛЬЗОВАННЫХ ИСТОЧНИКОВ}
    \section*{Список использованных источников} %\section*{СПИСОК ИСПОЛЬЗОВАННЫХ ИСТОЧНИКОВ}

    \renewcommand{\addcontentsline}[3]{}% Remove functionality of \addcontentsline
    \renewcommand{\section}[2]{}% Remove functionality of \section

    \begin{thebibliography}{}

      \bibitem{kubernetes_install}
      2-K8s - Поднятие простого Локального K8s Cluster на Windows - YouTube
      [Электронный ресурс]. -
      Режим доступа:
      \url{https://www.youtube.com/watch?v=WAIrMmCQ3hE}.
      Дата доступа: 28.02.2023.

      \bibitem{kubernetes_install_kubectl}
      Install and Set Up kubectl on Linux | Kubernetes
      [Electronic resource]. -
      Mode of access:
      \url{https://kubernetes.io/docs/tasks/tools/install-kubectl-linux/}.
      Date of access: 28.02.2023.

      \bibitem{kubernetes_install_minikube}
      minikube start | minikube
      [Electronic resource]. -
      Mode of access:
      \url{https://minikube.sigs.k8s.io/docs/start/}.
      Date of access: 28.02.2023.

      \bibitem{kubernetes_theory}
      7-K8s - Главные Объекты Kubernetes, из чего состоит K8s - Кубернетес на простом языке - YouTube
      [Электронный ресурс]. -
      Режим доступа:
      \url{https://www.youtube.com/watch?v=ouKaU1B5eKM&list=PLg5SS_4L6LYvN1RqaVesof8KAf-02fJSi&index=7}.
      Дата доступа: 28.02.2023.

      \bibitem{kubernetes_create_pod}
      8-K8s - Создание и Управление - PODS - Кубернетес на простом языке - YouTube
      [Электронный ресурс]. -
      Режим доступа:
      \url{https://www.youtube.com/watch?v=kGwe8IEDiX4&list=PLg5SS_4L6LYvN1RqaVesof8KAf-02fJSi&index=8}.
      Дата доступа: 28.02.2023.

    \end{thebibliography}
  \endgroup
  % = = = = = = = = = = = = = = = =
\end{document}
