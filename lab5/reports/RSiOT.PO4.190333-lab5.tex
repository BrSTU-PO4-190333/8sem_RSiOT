\documentclass[12pt, a4paper, simple]{eskdtext}

\usepackage{config/main.env}
\usepackage{config/report.env}
\usepackage{styles/listing}
\usepackage{styles/lists}
\usepackage{styles/SectionMargins}
\usepackage{styles/table}
\usepackage{styles/TableOfContent}
\usepackage{styles/url}

\begin{document}
  \input{include/TitlePage.report.tex}

  % = = = = = = = = = = = = = = = =
  \ESKDstyle{empty}
  \begin{center}
    \textbf{Отчёт лабораторной работы №\envReportLabNumber}
  \end{center}

  \paragraph{} \textbf{Тема}: <<\envReportTitle>>

  \paragraph{} \textbf{Цель}:
  Знакомство с Docker.

  \paragraph{} \textbf{Что нужно сделать}:

  Разверните приложения APP (MS2) и MS3 в Docker.

  % Задеплойте данные приложения в Compute Engine и App Engine соответственно.

  \paragraph{} \textbf{Ход работы}:

  \begin{center}
    \textbf{Сборка MS2-образа из Dockerfile и отправка на DockerHub}
  \end{center}

  \lstinputlisting[
    frame=single,
    rulecolor=\color{blue},
  ]{../sources/MS2/prod.Dockerfile}

  \begin{lstlisting}[
    frame=single,
    rulecolor=\color{blue},
    name={Команды bash по отправке MS2-образа Dockerfile на DockerHub},
  ]
docker login
docker build --file prod.Dockerfile -t pavelinnokentevichgalanin/8sem-rsiot-ms2 .
docker push pavelinnokentevichgalanin/8sem-rsiot-ms2
\end{lstlisting}

  \begin{center}
    \textbf{Сборка MS3-образа из Dockerfile и отправка на DockerHub}
  \end{center}

  \lstinputlisting[
    frame=single,
    rulecolor=\color{magenta},
  ]{../sources/MS3/prod.Dockerfile}

  \begin{lstlisting}[
    frame=single,
    rulecolor=\color{magenta},
    name={Команды bash по отправке MS3-образа Dockerfile на DockerHub},
  ]
docker login
docker build --file prod.Dockerfile -t pavelinnokentevichgalanin/8sem-rsiot-ms3 .
docker push pavelinnokentevichgalanin/8sem-rsiot-ms3
\end{lstlisting}

  \newpage
  \begin{center}
    \textbf{Образ MS2 из DockerHub в docker-compose.yml (на Compute Engine)}
  \end{center}

  \lstinputlisting[
    frame=single,
    rulecolor=\color{blue},
  ]{../sources/DOCKER/MS2/ssh.env.example}
  
  \lstinputlisting[
    frame=single,
    rulecolor=\color{blue},
  ]{../sources/DOCKER/MS2/Makefile}
  
  \lstinputlisting[
    frame=single,
    rulecolor=\color{blue},
  ]{../sources/DOCKER/MS2/ssh.docker-compose.yml}


  \newpage
  \begin{center}
    \textbf{Файл docker-compose.yml для MS3 на Google Cloud Compute Engine}
  \end{center}

  \lstinputlisting[
    frame=single,
    rulecolor=\color{magenta},
  ]{../sources/DOCKER/MS3/ssh.env.example}
  
  \lstinputlisting[
    frame=single,
    rulecolor=\color{magenta},
  ]{../sources/DOCKER/MS3/Makefile}
  
  \newpage

  \lstinputlisting[
    frame=single,
    rulecolor=\color{magenta},
  ]{../sources/DOCKER/MS3/ssh.docker-compose.yml}

\begin{lstlisting}[
  language=bash,
  frame=single,
  rulecolor=\color{green},
  name={Команды bash на Google Cloud Compute Engine},
]
sudo apt update
sudo apt install -y docker.io docker-compose make git

ssh-keygen # Enter, Enter, Enter
cat ~/.ssh/id_rsa.pub # Copy and paste to https://github.com/settings/ssh/new

git clone git@github.com:BrSTU-PO4-190333/8sem_RSiOT.git
cd 8sem_RSiOT/lab5/sources/DOCKER

cd MS2
make ssh-d      # Запустили MS2, Postgres и WatchTower
make ssh-logs   # Смотрим логи того как запускаются контейнеры

cd ../MS3
make ssh-d      # Запустили MS3 и WatchTower
make ssh-logs   # Смотрим логи того как запускаются контейнеры
\end{lstlisting}

  \paragraph{} \textbf{Вывод}:
  Собрали образ MS2 и MS3 из Dockerfile.
  Выложили образы на DockerHub \cite{dockerhub}.
  На Compute Engine установили docker.io, docker-compose, git.
  Склонировали репозиторий с docker-compose.yml.
  Запустили docker-compos.yml.
  Настроили Watchtower \cite{watchtower}, чтобы каждые 150 секунд следил за обновлениями на DockerHub, автоматически выкачивал новый образ,
  удалял старый образ и поднимал новый скаченный образ.

  % = = = = = = = = = = = = = = = =
  % \newpage
  \begingroup
    \phantomsection
    \addcontentsline{toc}{section}{СПИСОК ИСПОЛЬЗОВАННЫХ ИСТОЧНИКОВ}
    \section*{Список использованных источников} %\section*{СПИСОК ИСПОЛЬЗОВАННЫХ ИСТОЧНИКОВ}

    \renewcommand{\addcontentsline}[3]{}% Remove functionality of \addcontentsline
    \renewcommand{\section}[2]{}% Remove functionality of \section

    \begin{thebibliography}{}

      \bibitem{dockerhub}
      Docker Hub
      [Electronic resource]. -
      Mode of access:
      \url{https://hub.docker.com}.
      {\scriptsize
      Date of access: 27.02.2023.
      }

      \bibitem{watchtower}
      Arguments - Watchtower
      [Electronic resource]. -
      Mode of access:
      \url{https://containrrr.dev/watchtower/arguments}.
      Date of access: 27.02.2023.

    \end{thebibliography}
  \endgroup
  % = = = = = = = = = = = = = = = =
\end{document}
